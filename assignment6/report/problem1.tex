\section{Optimization}
\subsection*{a)}
An occurence of an expression E is called a common subexpression if E was previously computed and  the values of the variables in E have not changed since the previous computations. To perform a common subexpression elimination on the flow graph, the \emph{e}-assignment can be removed and {e} can be replaced by \emph{i}. \\
\subsection*{b)}
Copy Propagation is to use \emph{v} for \emph{u} whenever possible after the copy statement \emph{u = v}. Below is an example of B3 after copy propagation.\\
\begin{tabular}{|l|}
\hline
t = a * c	\\
i = i + a * c	\\
j = i		\\
d = 0		\\
d = i + 2	\\
e = i + t	\\
f = f * (i + t) \\
\hline
\end{tabular}

\subsection*{c)}
The "code motion" transformation takes an expression that yields the same result independent of the number os times a loop is executed and evaluates the expression outside the loop.\\
B1
\begin{tabular}{|l|}
\hline
i = 1		\\
t = a * c	\\
d = 0		\\
\hline
\end{tabular}
B3
\begin{tabular}{|l|}
\hline
i = i + t	\\
j = i		\\
d = j + 2	\\
e = i + t	\\
f = f * e	\\
\hline
\end{tabular}

\subsection*{c)}
Dead code is code that never gets used. Dead code is never introduced intentionally by the programmer, as it does not improve readabillity. A conditional statement that will always be false is an example of dead code. In the flow graph, the \emph{d = 0} statement will never be used for anything, removing it would be a \emph{dead code elimination}.

