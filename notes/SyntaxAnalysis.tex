\section{Syntax Analysis}
\emph{LL grammar:} Left to right, left recursive. \emph{LR grammar:} Left to right, right recursive. \\

\subsection{Context-Free Grammars}
CFG consists of:
\begin{enumerate}
	\item{\emph{terminals:} basic symbols from witch strings are formed. 			}
	\item{\emph{nonterminals:} syntactic variables that denotes sets of strings 	}
	\item{\emph{a start symbol:} nonterminal that denotes the language generated by the grammar. }
	\item{\emph{productions:} specifies the manner in witch the terminals and non terminals can be combined to form strings. 	}
	\begin{enumerate}
		\item{A nonterminal called the \emph{head} of the production. Defines some of the strings denoted by the head.}
		\item{The symbol $\rightarrow$ or ::=}
		\item{A \emph{body} consisting of zero or more terminals and nonterminals.}
	\end{enumerate}
\end{enumerate}

\emph{Ambiguity: } A grammar is ambigous when it can produce more than one leftmost derivation of or more than one rightmost derivation of the same sentence. If the grammar can not be made unambigous, it is prefferrable to have disambiguating rules that throws away undesirebla parse-trees.

\subsection{Writing a Grammar}
\emph{Left recursion:} A grammar is left recursive if it has a nonterminal A such that there is a derivation $A \rightarrow^+ A\alpha$ for some string $\alpha$. Top-down parsers cannot use a left-recursive grammar. \\
\emph{left factoring:} A grammar transformation used to produce a grammar suitable for predictive or top-down parsing. Done by finding the longest prefix $\alpha$ common to two or more of its alternative. Add a new production with \emph{remaining}|$\epsilon$. \\
